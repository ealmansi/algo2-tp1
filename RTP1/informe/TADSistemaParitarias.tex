% \emph{
%   Para generar una instancia de SistemaParitarias es necesario proveer la informaci\'on est\'atica del mismo a trav\'es de una instancia de DatosSistemaParitarias. Despu\'es, la din\'amica de las paritarias se maneja (re)abriendo paritarias mediante el generador AbrirParitaria, o registrando el acontecimiento de una reuni\'on mediante el generador HuboReunion. Las operaciones empresasNegociando y gremiosNegociando permiten conocer las empresas y gremios con paritarias abiertas, y la operaci\'on gremioInconformista permite conocer alguno de los gremios que m\'as veces reabri\'o paritarias.
% }

\begin{tad}{\tadNombre{SistemaParitarias}}

% \tadIgualdadObservacional{si}{si'}{sistema}{
%   gremios(si) $\igobs$ gremios(si') \mand \\
%   ((\paratodoC{ng}{nombre}{ ((\existeC{g}{gremio}{ g \en gremios(si)}) nombre(g) = ng})) aliados(ng, si) $\igobs$ aliados(ng, si') \mand historial(ng, si) $\igobs$ historial(ng, si'))
% }

\tadGeneros{sistema}

% \tadExporta{sistema, observadores, generadores, gremiosNegociando, empresasNegociando, \#trabajadoresNegociando, gremioInconformista}

% \tadUsa{\tadNombre{Bool}, \tadNombre{Nat}, \tadNombre{Datos}, \tadNombre{Historial}, \tadNombre{Conjunto($\alpha$)}}

\tadObservadores
	\tadAlinearFunciones{paritaria}{gremio/g, sistema/si}
  \tadOperacion{gremios}{sistema/si}{conj(gremio)}{}
  \tadOperacion{aliados}{gremio/g, sistema/si}{conj(gremio)}{g \en gremios(si)}
  \tadOperacion{paritaria}{gremio/g, sistema/si}{paritaria}{g \en gremios(si)}

\tadGeneradores
	\tadAlinearFunciones{HuboReunion}{reunion/rn, sistema/si}
  \tadOperacion{CrearSistema}{confederacion/c}{sistema}{}
  \tadOperacion{AbrirParitaria}{gremio/g, sistema/si}{sistema}{ g \en gremios(si) \mandl \mnot estaAbierta?(paritaria(g,si)) }
  \tadOperacion{HuboReunion}{reunion/rn, sistema/si}{sistema}{ gremio(rn) \en gremios(si)\mandl (\paratodoC{ga}{gremio}{ga \en aliados(gremio(rn),si)})\\ gremioPidio(rn) > maximoAumento(paritaria(ga,si))}

\tadOtrasOperaciones
  \tadAlinearFunciones{\#trabajadoresNegociando}{sistema/si}
  \tadOperacion{gremiosNegociando}{sistema/si}{conj(gremio)}{}
  \tadOperacion{empresasNegociando}{sistema/si}{conj(empresa)}{}
  \tadOperacion{\#trabajadoresNegociando}{sistema/si}{nat}{}
  \tadOperacion{gremioInconformista}{sistema/si}{gremio}{}
  
  \vspace{2mm}
  \tadAlinearFunciones{hayQueReabrirParitaria}{gremio/g,reunion/rn,sistema/si}
  \tadOperacion{hayQueReabrirParitaria}{gremio/g,reunion/rn,sistema/si}{bool}{g \en gremios(si)\mand gremio(rn) \en gremios(si)}

  \vspace{2mm}
  \tadAlinearFunciones{obtenerGremiosNegociando}{conj(gremio)/cg,sistema/si}
  \tadOperacion{obtenerGremiosNegociando}{conj(gremio)/cg,sistema/si}{conj(gremio)}{cg \incluido gremios(si)}
  \tadOperacion{obtenerEmpresas}{conj(gremio)/cg}{conj(empresa)}{}
  \tadOperacion{contarTrabajadores}{conj(gremio)/cg}{nat}{}

  \vspace{2mm}
  \tadAlinearFunciones{obtenerGremiosInconformistas}{conj(gremio)/cg,sistema/si}
  \tadOperacion{obtenerGremiosInconformistas}{conj(gremio)/cg,sistema/si}{conj(gremio)}{cg \incluido gremios(si)}
  \tadOperacion{esInconformista}{gremio/g,sistema/si}{bool}{g \en gremios(si)}
  \tadOperacion{\#reaperturasMaxima}{conj(gremio)/cg,sistema/si}{nat}{cg \incluido gremios(si)}

\tadAxiomas{\paratodo{c}{confederacion}, \paratodo{g}{gremio}, \paratodo{rn}{reunion}, \paratodo{cg}{conj(gremio)}, \paratodo{si}{sistema} } \\

	\tadAlinearAxiomas{gremios(AbrirParitaria(g,si))}
  \tadAxioma{gremios(CrearSistema(c))}{gremios(c)}
  \tadAxioma{gremios(AbrirParitaria(g,si))}{gremios(si)}
  \tadAxioma{gremios(HuboReunion(rn,si))}{gremios(si)}

  \vspace{2mm}
	\tadAlinearAxiomas{aliados(g1,AbrirParitaria(g2,si))}
  \tadAxioma{aliados(g,CrearSistema(c))}{aliados(g, c)}
  \tadAxioma{aliados(g1,AbrirParitaria(g2,si))}{aliados(g1, si)}
  \tadAxioma{aliados(g,HuboReunion(rn,si))}{aliados(g, si)}

  \vspace{2mm}
  \tadAlinearAxiomas{paritaria(g1,AbrirParitaria(g2,si))}
  \tadAxioma{paritaria(g,CrearSistema(c))}{CrearParitaria}
  \tadAxioma{paritaria(g1,AbrirParitaria(g2,si))}{\iffi{g1 = g2}
                                            {Abrir(paritaria(g1,si))}
                                            {paritaria(g1,si)}}
  \tadAxioma{paritaria(g,HuboReunion(rn,si))}{\iffi{g = gremio(rn)}
                                            {Reunion(paritaria(g,si))}
                                            {\iffiI{hayQueReabrirParitaria(g,rn,si)}
                                              {Abrir(paritaria(g,si))}
                                              {paritaria(g,si)}}}

  \vspace{2mm}
  \tadAlinearAxiomas{\#trabajadoresNegociando(si)}
  \tadAxioma{gremiosNegociando(si)}{obtenerGremiosNegociando(gremios(si),si)}
  \tadAxioma{empresasNegiociando(si)}{obtenerEmpresas(gremiosNegociando(si))}
  \tadAxioma{\#trabajadoresNegociando(si)}{contarTrabajadores(gremiosNegociando(si))}
  \tadAxioma{gremioInconformista(si)}{dameUno(obtenerGremiosInconformistas(gremios(si),si))}

  \vspace{2mm}
  \tadAlinearAxiomas{hayQueReabrirParitaria(g,rn,si)}
  \tadAxioma{hayQueReabrirParitaria(g,rn,si)}{\mnot estaAbierta?(paritaria(g,si)) \mand \\ g \en aliados(gremio(rn),si) \mand huboAcuerdo(rn) \mandl \\ aumento(rn) \mgt maximoAumento(paritaria(g,si))}

  \vspace{2mm}
  \tadAlinearAxiomas{obtenerGremiosNegociando(cg,si)}
  \tadAxioma{obtenerGremiosNegociando(cg,si)}{\iffi{\esvacio{cg}}
                                        {\vacio}
                                        {\iffiI{estaAbierta?(paritaria(dameUno(cg),si))}
                                        {Ag(dameUno(cg),obtenerGremiosNegociando(sinUno(cg),si))}
                                        {obtenerGremiosNegociando(sinUno(cg),si)}}}
  
  \tadAlinearAxiomas{contarTrabajadores(cg)}
  \tadAxioma{obtenerEmpresas(cg)}{\iffi{\esvacio{cg}}
                            {\vacio}
                            {Ag(empresas(dameUno(cg)), obtenerEmpresas(sinUno(cg)))}}
  \tadAxioma{contarTrabajadores(cg)}{\iffi{\esvacio{cg}}
                                {0}
                                {\#(trabajadores(dameUno(cg))) + contarTrabajadores(sinUno(cg))}}

  \vspace{2mm}
  \tadAlinearAxiomas{obtenerGremiosInconformistas(cg,si)}
  \tadAxioma{obtenerGremiosInconformistas(cg,si)}{\iffi{\esvacio{cg}}
                                            {\vacio}
                                            {\iffiI{esInconformista(dameUno(cg),si)}
                                            {Ag(dameUno(cg),obtenerGremiosInconformistas(sinUno(cg),si))}
                                            {obtenerGremiosInconformistas(sinUno(cg),si)}}}
  \tadAlinearAxiomas{esInconformista(g,si)}
  \tadAxioma{esInconformista(g,si)}{\#reaperturasMaxima(gremios(si),si) \mleq \#reaperturas(paritaria(g,si))}
  \tadAlinearAxiomas{\#reaperturasMaxima(cg,si)}
  \tadAxioma{\#reaperturasMaxima(cg,si)}{\iffi{\esvacio{cg}}
                                {0}
                                {max(\#reaperturas(paritaria(dameUno(cg),si))), \#reaperturasMaxima(sinUno(cg),si))}}

\end{tad}